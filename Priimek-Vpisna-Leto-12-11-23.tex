\documentclass[mat1, tisk]{fmfdelo}
\usepackage{graphicx}
\usepackage{enumitem}
% \documentclass[fin1, tisk]{fmfdelo}
% Če pobrišete možnost tisk, bodo povezave obarvane,
% na začetku pa ne bo praznih strani po naslovu, …

%%%%%%%%%%%%%%%%%%%%%%%%%%%%%%%%%%%%%%%%%%%%%%%%%%%%%%%%%%%%%%%%%%%%%%%%%%%%%%%
% METAPODATKI
%%%%%%%%%%%%%%%%%%%%%%%%%%%%%%%%%%%%%%%%%%%%%%%%%%%%%%%%%%%%%%%%%%%%%%%%%%%%%%%

% - vaše ime
\avtor{Tadeja Možina}

% - naslov dela v slovenščini
\naslov{Metrična dimenzija grafa deliteljev niča}

% - naslov dela v angleščini
\title{Metric dimension of a zero-divisor graph}

% - ime mentorja/mentorice s polnim nazivom:
%   - doc.~dr.~Ime Priimek
%   - izr.~prof.~dr.~Ime Priimek
%   - prof.~dr.~Ime Priimek
%   za druge variante uporabite ustrezne ukaze
\mentor{izr.~prof.~dr.~David Dolžan}

% - leto diplome
\letnica{2024} 

% - povzetek v slovenščini
%   V povzetku na kratko opišite vsebinske rezultate dela. Sem ne sodi razlaga
%   organizacije dela, torej v katerem razdelku je kaj, pač pa le opis vsebine.
\povzetek{V diplomski nalogi preučujemo metrično dimenzijo grafa deliteljev niča.}

% - povzetek v angleščini
\abstract{In this thesis, we study the metric dimension of a zero-divisor graph.}

% - klasifikacijske oznake, ločene z vejicami
%   Oznake, ki opisujejo področje dela, so dostopne na strani https://www.ams.org/msc/
\klasifikacija{74B05, 65N99}

% - ključne besede, ki nastopajo v delu, ločene s \sep
\kljucnebesede{naravni logaritem\sep nenaravni algoritem}

% - angleški prevod ključnih besed
\keywords{natural logarithm\sep unnatural algorithm} % angleški prevod ključnih besed

% - angleško-slovenski slovar strokovnih izrazov
\slovar{
\geslo{continuous}{zvezen}
\geslo{uniformly continuous}{enakomerno zvezen}
\geslo{compact}{kompakten -- metrični prostor je kompakten, če ima v njem vsako zaporedje stekališče; podmnožica evklidskega prostora je kompaktna natanko tedaj, ko je omejena in zaprta  }
\geslo{glide reflection}{zrcalni zdrs ali zrcalni pomik -- tip ravninske evklidske izometrije, ki je kompozitum zrcaljenja in translacije vzdolž iste premice}  
\geslo{lattice}{mreža}  
\geslo{link}{splet}
\geslo{partition}{\textbf{$\sim$ of a set} razdelitev množice; \textbf{$\sim$ of a number} razčlenitev števila}
}

% - ime datoteke z viri (vključno s končnico .bib), če uporabljate BibTeX
\literatura{literatura.bib}

%%%%%%%%%%%%%%%%%%%%%%%%%%%%%%%%%%%%%%%%%%%%%%%%%%%%%%%%%%%%%%%%%%%%%%%%%%%%%%%
% DODATNE DEFINICIJE
%%%%%%%%%%%%%%%%%%%%%%%%%%%%%%%%%%%%%%%%%%%%%%%%%%%%%%%%%%%%%%%%%%%%%%%%%%%%%%%

% naložite dodatne pakete, ki jih potrebujete
\usepackage{algpseudocode}  % za psevdokodo
\usepackage{algorithm}      % za algoritme
\floatname{algorithm}{Algoritem}
\renewcommand{\listalgorithmname}{Kazalo algoritmov}

% deklarirajte vse matematične operatorje, da jih bo LaTeX pravilno stavil
% \DeclareMathOperator{\conv}{conv}
% na razpolago so naslednja matematična okolja, ki jih kličemo s parom
% \begin{imeokolja}[morebitni komentar v oklepaju] ... \end{imeokolja}
%
% definicija, opomba, primer, zgled, lema, trditev, izrek, posledica, dokaz

% za številske množice uporabite naslednje simbole
\newcommand{\R}{\mathbb R}
\newcommand{\N}{\mathbb N}
\newcommand{\Z}{\mathbb Z}
% Lahko se zgodi, da je ukaz \C definiral že paket hyperref,
% zato dobite napako: Command \C already defined.
% V tem primeru namesto ukaza \newcommand uporabite \renewcommand
\newcommand{\C}{\mathbb C}
\newcommand{\Q}{\mathbb Q}

%%%%%%%%%%%%%%%%%%%%%%%%%%%%%%%%%%%%%%%%%%%%%%%%%%%%%%%%%%%%%%%%%%%%%%%%%%%%%%%
% ZAČETEK VSEBINE
%%%%%%%%%%%%%%%%%%%%%%%%%%%%%%%%%%%%%%%%%%%%%%%%%%%%%%%%%%%%%%%%%%%%%%%%%%%%%%%

\begin{document}

\section{Uvod}

%Na začetku prvega poglavja/razdelka (ali v samostojnem razdelku z naslovom
%Uvod) napišite kratek zgodovinski in matematični uvod. Pojasnite motivacijo za
%problem, kje nastopa, kje vse je bil obravnavan. Na koncu opišite tudi
%organizacijo dela -- kaj je v kakšnem razdelku.

%Če se uvod naravno nadaljuje v besedilo prvega poglavja, lahko nadaljujete z
%besedilom v istem razdelku, sicer začnete novega. Na začetku vsakega
%razdelka/podraz\-delka poveste, čemu se bomo posvetili v nadaljevanju. Pri
%pisanju uporabljajte ukaze za matematična okolja, med formalnimi enotami
%dodajte vezno razlagalno besedilo.
Pojem metrične dimenzije grafa je v sedemdesetih letih prejšnjega stoletja 
uvedel Peter J. Slater, problem iskanja le te pa sta kot prva raziskovala 
Frank Harary in Robert Melter~\cite{7dolzan}~\cite{4pirzada17}. Uporablja se 
na različnih področjih, kot na primer v farmacevtski kemiji, navigaciji 
robotov in kombinatorični optimizaciji~\cite{0OuSh}.
%
%
\section{1}
V tem poglavju bomo predstavili osnovne pojme, povezane z metrično dimenzijo grafa 
deliteljev niča. Pri tem bomo sledili~\cite{0OuSh}. Skozi celotno diplomsko nalogo 
bodo bili vsi grafi neusmerjeni.
%
\subsection{Osnovne definicije}
%
\begin{definicija}
  Naj bo R kolobar in $Z(R)$ njegova množica deliteljev niča.
  \emph{Graf deliteljev niča kolobarja R} je enostaven neusmerjen graf z množico
  vozlišč $Z^{*}(R) = Z(R)\setminus\{0\} $, kjer sta dve različni vozlišči $a,b \in R $
  povezani natanko tedaj, ko je $ab = 0$ ali $ba = 0$. Označimo ga z $\Gamma(R)$.
\end{definicija}
%
\begin{zgled}\label{prim1}
  Poglejmo si graf deliteljev niča kolobarja $\Z_{10}$. Množica njegovih vozlišč je 
  $V(\Gamma(\Z_{10})) = Z^{*}(\Z_{10}) = \{2, 4, 5, 6, 8\}$, množica povezav pa 
  je enaka $E(\Gamma(\Z_{10})) = \{2 - 5, 4 - 5, 5 - 6, 5 - 8\}$.
  \begin{figure}[H]
    \centering
    \includegraphics[scale=0.5]{z10.png}
    \caption{Graf deliteljev niča $\Z_{10}$}
  \end{figure}
\end{zgled}
%
\begin{zgled}\label{zgled2.3}
  Poglejmo si še graf deliteljev niča kolobarja $\Z_{12}$. Množica njegovih vozlišč je 
  $V(\Gamma(\Z_{12})) = Z^{*}(\Z_{12}) = \{2, 3, 4, 6, 8, 9, 10\}$, povezave pa so 
  prikazane na sliki.
  \begin{figure}[H]
    \centering
    \includegraphics[scale=0.5]{z12.png}
    \caption{Graf deliteljev niča $\Z_{12}$}
  \end{figure}
\end{zgled}
%
Spomnimo se še definicije razdalje med dvema vozliščema v grafu. Za poljubni vozlišči 
$u,v \in V(\Gamma)$ je \emph{razdalja} med $u$ in $v$, označena z $d(u,v)$, dolžina 
najkrajše poti med njima. Če ne obstaja pot med $u$ in $v$, je $d(u,v) = \infty $.
Z uporabo te definiramo predstavitev $v$ glede na $W$.
%
\begin{definicija}
  Naj bo $W = \{ w_1,w_2, \ldots, w_k \}$ urejena podmnožica od $V(\Gamma)$ in 
  $v \in V(\Gamma)$. Potem 
  se $k$-dimenzionalni vektor $r(v|W)=( d(v,w_1), \ldots, d(v,w_k) )$ imenuje 
  \emph{predstavitev $v$ glede na $W$}. 

  Rečemo, da je $v$ \emph{rešljiv z W}, če velja $r(v|W) \neq r(u|W)$, 
  za vsak $u \in V(\Gamma)$ različen od $v$, torej 
  $( d(v,w_1), \ldots, d(v,w_k) ) \neq ( d(u,w_1), \ldots, d(u,w_k) )$ za vsak 
  $u \neq v$.
\end{definicija}
%
Množici $W$ pravimo \emph{rešljiva množica od $\Gamma$}, če imata poljubni različni 
vozlišči v $V(\Gamma)$ različni predstavitvi glede na $W$. Če je $W$ rešljiva množica 
z najmanjšo močjo, ji pravimo \emph{baza od $\Gamma$}.
%
%Metrična dimenzija
\begin{definicija}
  \emph{Metrična dimenzija grafa $\Gamma$} je moč njene baze. Označimo jo z $dim(\Gamma)$.
\end{definicija}
%
\begin{opomba}\label{opomba2.7}
  %dovolj je gledati predstavitve elementov izven W glede na W.
  Če preverjamo, ali je $W= \{w_1,w_2, \ldots, w_k\}$ podmnožica od $V(\Gamma)$ 
  rešljiva množica $\Gamma$, je dovolj gledati predstavitve elementov iz 
  $V(\Gamma) \setminus W$ glede na $W$. Namreč če vzamemo $w_i \in W$ bo 
  $(d(w_i,w_1),d(w_i,w_2), \ldots d(w_i,w_k))$ na $i$-tem mestu imela 0. 
  Če bi katerokoli drugo vozlišče $u$ imelo enako predstavitev kot $w_i$, bi 
  potem moralo veljati, da $d(w_i,w_i) = d(u,w_i)$, kar pa je edino možno, 
  če je $u = w_i$.
\end{opomba}
%
\begin{opomba}\label{opomba2.8}
  Očitno je vsaka nadmnožica rešljive množice grafa $\Gamma$ tudi sama rešljiva 
  množica. Res, naj bo $W_2$ nadmnožica rešljive množice $W_1$ grafa $\Gamma$, 
  torej $W_1 \subseteq W_2$. Po opombi \ref{opomba2.7} je dovolj gledati 
  predstavitve elementov izven $W_2$. Ker so predstavitve teh elementov glede na 
  $W_1$ različne in $W_2$ vsebuje $W_1$, bodo njihove predstavitve glede na 
  $W_2$ bile enake tistim v $W_1$ z nekaj dodanimi komponentami, torej še vedno 
  različne.
\end{opomba}
%
\begin{zgled}\label{prim1.2}
  Poiščimo metrično dimenzijo grafa deliteljev niča kolobarja $\Z_{10}$ iz 
  primera \ref{prim1}. Razdalje 
  med dvema poljubnima različnima vozliščema so 1 ali 2. Če za $W = \{w\}$ vzamemo 
  katerokoli enoelementno množico, bosta obstajali vsaj dve vozlišči iz 
  $\{2,4,6,8\}$, ki imata enako razdaljo do $w$, torej to ni baza. Podoben razmislek 
  naredimo, če je $W$ dvoelementna množica - v $\{2,4,6,8\}$ bosta obstajali dve 
  vozlišči z enako predstavitvijo glede na $W$. Poglejmo sedaj, če obstaja rešljiva 
  množica moči 3. Vzemimo $W = \{2,4,6\}$. Edina dva elementa, ki ju rabimo pogledati 
  sta 5 in 8. Izračunamo 
  $(d(5,2),d(5,4),d(5,6)) = (1,1,1) \neq (d(8,2),d(8,4),d(8,6)) = (2,2,2)$. Torej je 
  metrična dimenzija $\Gamma(\Z_{10})$ = 3.
\end{zgled}
%
\begin{opomba}
  Graf $\Gamma$ ima lahko več rešljivih množic iste moči. Pri zgledu \ref{prim1.2} 
  bi za rešljivo množico moči 3 (in torej tudi bazo) tako lahko vzeli tudi 
  $\{2,4,8\}$, $\{2,6,8\}$ ali $\{4,6,8\}$.
\end{opomba}
%
\subsection{Meje metrične dimenzije grafa}
%
Ponovimo nekaj pojmov, ki jih bomo potrebovali v nadaljevanju. Za graf 
$\Gamma = (V, E)$ je njegov \emph{premer}, označen z $diam(\Gamma)$, največja 
razdalja med dvema vozliščema v grafu. \emph{Soseščina} vozlišča $v$ je 
množica $N(v) = \{ x \in V(\Gamma)~|~x \sim v \}$, torej množica vozlišč 
s katerimi ima $v$ povezavo. 

Navedimo sedaj prvo trditev, ki postavi meje metrične dimenzije grafa~\cite{10chartrand}.
%
\begin{trditev}
  Naj bo $\Gamma$ povezan graf z $n$ vozlišči in premerom $d$ = $diam(\Gamma)$. Potem velja
  $n - d^{dim(\Gamma)} \leq dim(\Gamma) \leq n - d$.
\end{trditev}
\begin{dokaz}
  Poglejmo si najprej drugo neenakost. Naj bosta $u$ in $v$ vozlišči grafa 
  $\Gamma$, pri katerih je razdalja največja možna, torej $d(u,v) = d$. 
  Njuno $uv$-pot dolžine $d$ zapišemo kot $u=v_0, v_1, \ldots, v_d=v$. 
  Naj bo $W = V(\Gamma) \setminus \{v_1, \ldots, v_d \}$. Potem je 
  $d(u,v_i) = i$ za $i = 1, \ldots, d$. Ker je $u \in W$ in imajo 
  vsa vozlišča grafa izven $W$ paroma različno razdaljo do $u$ in 
  torej različno predstavitev glede na $W$, je $W$ 
  rešljiva množica od $\Gamma$.

  Za dokaz prve neenakosti izberimo bazo $B$ od $\Gamma$ moči $k$. 
  Poglejmo vozlišča iz $V(\Gamma) \setminus B$, teh je $n-k$. 
  Njihove predstavitve 
  glede na $B$ so vektorji dolžine $k$, katerih koordinate so števila med 
  $1$ in $d$. Vsakemu vozlišču iz $V(\Gamma) \setminus B$ priredimo 
  tak vektor. Ker je $B$ baza, morajo biti vektorji med sabo različni. 
  Torej je preslikava injektivna in dobimo $n - k \leq d^k$. 
  Neenakost preuredimo in dobimo željen rezultat.
\end{dokaz}
%
Navedimo še nekaj novih definicij, s katerimi si bomo pomagali še dodatno 
izboljšati meje metrične dimenzije grafa.
%
\begin{definicija}
  Za vozlišči $u, v\in V(\Gamma)$ pravimo, da sta \emph{dvojčka}, če velja 
  $N(u) \cup \{u\} = N(v) \cup \{v\} $, ko $u \sim v$, ali 
  $N(u) = N(v) $, ko $u \nsim v$.

  Množici vseh dvojčkov vozlišča $v$ rečemo \emph{množica dvojčkov vozlišča $v$}. 
  Označimo jo z $Tw(v)$.
\end{definicija}
%
Opazimo, da $V(\Gamma)$ lahko zapišemo kot disjunktno unijo množic 
dvojčkov, saj vsaka množica dvojčkov $Tw(v)$ vsebuje vsaj en element, vozlišče $v$.
Glede na to, koliko vozlišč vsebujejo, množice dvojčkov delimo v dve skupini.
%
\begin{definicija}
  Množici dvojčkov $Tw(v)$ pravimo, da je \emph{prava množica dvojčkov grafa $\Gamma$}, 
  če je $|Tw(v)| \geq 2$. Če je $|Tw(v)| = 1$, ji rečemo 
  \emph{osamljena množica dvojčkov grafa $\Gamma$}. 

  Število pravih množic grafa $\Gamma$ označimo z \emph{$\alpha(\Gamma)$}, število 
  osamljenih množic dvojčkov pa z \emph{$\beta(\Gamma)$}.
\end{definicija}
%
\begin{zgled}
  Poglejmo množice dvojčkov iz zgleda \ref{zgled2.3}. Množice dvojčkov so 
  $\{9,3\},$ $\{8,4\},\{2,10\}$ in $\{6\}$. Število pravih množic grafa $\Gamma(\Z_{12})$ 
  je torej $\alpha(\Gamma(\Z_{12})) = 3$, število osamelih pa $\beta(\Gamma(\Z_{12})) = 1$.
\end{zgled}
%
Navedimo sedaj trditev, ki uporabi množice dvojčkov za omejitev metrične 
dimenzije grafa~\cite{3pirzada14}.
%
\begin{trditev}
  Naj bo $\Gamma$ povezan graf in $T_1, T_2, \ldots, T_{\alpha(\Gamma)}$ različne 
  prave množice dvojčkov grafa $\Gamma$. Potem velja:
  \begin{enumerate}[label=(\roman*)]
    \item Vsaka baza od $\Gamma$ vsebuje vsa, razen morda enega, vozlišča iz 
          vsakega $T_i$, za $1 \leq i \leq \alpha(\Gamma)$,
    \item Obstaja baza $P$ od $\Gamma$, da $P$ vsebuje največ $|T_i| - 1$ vozlišč 
          iz vsakega $T_i$, za $1 \leq i \leq \alpha(\Gamma)$,
    \item $|V(\Gamma)| - \alpha(\Gamma) - \beta(\Gamma) \leq dim(\Gamma) \leq |V(\Gamma)| - \alpha(\Gamma)$.
  \end{enumerate}
\end{trditev}
\begin{dokaz}
  Dokažimo najprej točko i). Naj bo $W$ rešljiva množica od $\Gamma$. Recimo, da trditev ne velja, 
  torej da obstajata vozlišči $u,v$ v isti pravi množici dvojčkov, ki nista obadva v $W$. 
  Naj bo ta množica dvojčkov $T_i$. Ker sta $u$ in $v$ v isti množici dvojčkov, 
  imata enake sosede, torej je njuna predstavitev glede na $W$ enaka, kar pa ni 
  možno, saj je $W$ baza.

  Dokažimo sedaj točko ii). Naj bo $B$ rešljiva množica od $\Gamma$ in 
  $T_1, \ldots, T_{\alpha(\Gamma)}$ njene prave množice dvojčkov ter 
  $T_{\alpha(\Gamma)+1}, \ldots, T_{\alpha(\Gamma)+\beta(\Gamma)}$ osamljene množice dvojčkov.
  Poglejmo množico $T_1$. 
  Po točki i) $B$ vsebuje vsa, razen morda enega, vozlišča iz $T_1$. Izberimo $W$ 
  rešljivo množico tako, da bo v celoti vsebovala $T_1$ - če $B$ že v celoti vsebuje 
  $T_1$, je $W = B$, če ne, pa za $W$ izberemo $B$, ki mu dodamo manjkajoče vozlišče in po 
  opombi \ref{opomba2.8} bo ta še vedno rešljiva množica. Naj bo $x \in T_1$ in 
  $W = \{x, w_1, w_2, \ldots\}$. Označimo $P_1 = W \setminus \{x\} = \{w_1, w_2, \ldots\}$ 
  in dokažimo, da je rešljiva množica.
  Vzemimo različni vozlišči $v_1, v_2 \in V(\Gamma)$. Če je vsaj eden izmed njiju v $P_1$, 
  že vemo, da sta njuni predstavitvi glede na $P_1$ različni. Naj bosta sedaj 
  $v_1, v_2 \notin P_1$. Recimo, da je njuna predstavitev 
  glede na $P_1$ enaka, torej $r(v_1|P_1) = r(v_2|P_1)$. Ker je $W$ rešljiva množica, 
  je njuna predstavitev glede na $W$ različna, $r(v_1|W) \neq r(v_2|W)$. Iz 
  tega sledi, da se $r(v_1|P_1)$ in $r(v_2|P_1)$ razlikujeta na prvi komponenti, 
  torej da je $d(x,v_1) \neq d(x,v_2)$. $T_1$ je prava množica dvojčkov, 
  torej obstaja $z \in T_1 $, da $z \neq x$. Ker je $z$ tudi v $P_1$ in 
  $r(v_1|P_1) = r(v_2|P_1)$, sledi, da $d(z,v_1) = d(z,v_2)$. Vendar pa sta 
  $x$ in $z$ v isti pravi množici dvojčkov, torej velja, da 
  $d(x,v_1) = d(z,v_1) = d(z,v_2) = d(x,v_1)$, iz česar pridemo v protislovje. 
  Torej je $P_1$ rešljiva množica grafa $\Gamma$. Postopek ponovimo še za 
  ostale prave množice dvojčkov $T_i$, za $2 \leq i \leq \alpha(\Gamma)$: 
  na $i$-tem koraku množici $P_i$ dopolnimo za največ eno vozlišče tako, da bo 
  v celoti vsebovala $T_i$, nato pa eno (ne nujno isto) vozlišče odstranimo. 
  Ker so prave množice dvojčkov $T_1, \ldots, T_{\alpha(\Gamma)}$ med seboj disjunktne, 
  je ta postopek smiseln.
  Na koncu bo bila $P_{\alpha(\Gamma)}$ rešljiva množica.

  Nazadnje dokažimo še točko iii). Po točki ii) obstaja baza $P$, ki ima iz vsake 
  množice dvojčkov $T_i$, $1 \leq i \leq \alpha(\Gamma)$ največ $|T_i| - 1$ vozlišč. 
  Torej $P$ je baza, ki ima največ $|V(\Gamma)| - \alpha(\Gamma)$ vozlišč in 
  posledično je $dim(\Gamma) \leq |V(\Gamma)| - \alpha(\Gamma)$. Spodnjo mejo dokažimo s 
  pomočjo točke i). Naj bodo $T_1, \ldots, T_{\alpha(\Gamma)}$ prave množice dvojčkov ter 
  $T_{\alpha(\Gamma)+1}, \ldots, T_{\alpha(\Gamma)+\beta(\Gamma)}$ osamljene množice dvojčkov. 
  Po točki i) vsaka baza vsebuje vse, razen morda enega, vozlišča iz 
  $T_1, \ldots, T_{\alpha(\Gamma)}$. Če upoštevamo, da je osamljenih množic $\beta(\Gamma)$, 
  dobimo $|V(\Gamma)| - \alpha(\Gamma) - \beta(\Gamma) \leq dim(\Gamma)$.
\end{dokaz}
%
Poglejmo definicijo določitvenega števila.
%
\begin{definicija}
  Naj bo $\Gamma$ graf z množico vozlišč $V(\Gamma)$. Podmnožica D od $V(\Gamma)$ 
  se imenuje \emph{določitvena množica od $\Gamma$}, če je edini avtomorfizem na 
  $\Gamma$, ki fiksira vsak element v D, identiteta.\\
  Velikosti najmanjše take podmnožice D od $V(\Gamma)$ pravimo 
  \emph{določitveno število $\Gamma$} in ga označimo s $fix(\Gamma)$.
\end{definicija}
%
Označimo še množico 
$O_v = \{ \phi(v) \in V(\Gamma)~|~ \phi \in Aut(\Gamma) \}$ kot 
\emph{orbito} vozlišča $v$ in $O_v = \{ \phi(v) \in V(\Gamma)~|~ \phi \in Aut(\Gamma) \}$
Navedimo trditev, ki velja za določitveno število grafa~\cite{1erwin}.
%
\begin{trditev}
  Naj bo $\Gamma$ graf z netrivialno grupo avtomorfizmov. Potem velja, da 
  je $fix(\Gamma)=1$ natanko tedaj, ko obstaja vozlišče $v \in V(\Gamma)$, 
  da je $|O_v| = |Aut(\Gamma)|$.
\end{trditev}









\begin{izrek}\label{izr:enakomerno}
Zvezna funkcija na zaprtem intervalu je enakomerno zvezna.
\end{izrek}

\begin{dokaz}
Na začetku dokaza, če je to le mogoče in smiselno, razložite idejo dokaza.

Dokazovali bomo s protislovjem. Pomagali si bomo z definicijo zveznosti in s
kompaktnostjo intervala.  Izberimo $\varepsilon>0$. Če $f$ ni enakomerno
zvezna, potem za vsak $\delta>0$ obstajata $x, y$, ki zadoščata
\begin{equation}\label{eq:razlika}
  |x-y|<\delta\quad \text{in}\quad |f(x)-f(y)| \ge \varepsilon. \qedhere
\end{equation}
\end{dokaz}

V zgornjem primeru smo kvadratek za konec dokaza postavili v zadnjo vrstico
besedila, ki je vrstična formula, s pomočjo ukaza \verb|\qedhere|.  Ta ukaz
ustrezno deluje znotraj okolij \emph{equation, align*} in podobnih, ne pa
znotraj \verb|$$ ... $$|.

Oglejmo si še enkrat neenačbi~\eqref{eq:razlika}. Na formule se sklicujemo z
ukazom \verb|\eqref{oznaka}|, ki postavi zaporedno številko enačbe
v oklepaje, na trditve in ostale enote pa z ukazom \verb|\ref{oznaka}|. Črni
pravokotnik ob robu strani označuje predolgo vrstico, kjer \LaTeX ni uspel
pravilno postaviti besedila, zato mu morate pomagati, npr.\ tako, da stavek
nekoliko preoblikujete, sami razdelite nedeljivo enoto (npr. razdelite
matematično formulo na dva dela) ali pa ponudite možnosti za deljenje težavne
besede s pomočjo znakov\verb|\-|, ki jih postavite na mesto, kjer se besedo sme
deliti, npr.\  \verb|te\-žav\-nost\-ni\-ca|. Zgoraj bi tako lahko zapisali:

Oglejmo si še enkrat neenačbi~\eqref{eq:razlika}. Za sklicevanje na označene
enote besedila imamo na razpolago dva ukaza; na formule se sklicujemo z ukazom
\verb|\eqref{oznaka}|, \dots

V predzadnjem odstavku je v oklepaju za okrajšavo npr.\ nastal predolg
presledek sredi stavka, saj je \LaTeX zaradi pike sklepal, da je na tem mestu
stavka konec. Tak predolg presledek preprečimo tako, da za piko sredi stavka
postavimo poševnico in za njo presledek, torej \verb|\ |.

Če dokaz trditve ne sledi neposredno formulaciji trditve, moramo povedati, kaj
bomo dokazovali. To naredimo tako, da ob ukazu za okolje dokaz dodamo neobvezni
parameter,  v katerem napišemo besedilo, ki se bo izpisalo namesto besede
\emph{Dokaz}, npr.\ \verb|\begin{Dokaz}[Dokaz izreka \ref{izr:enakomerno}]|.

\begin{dokaz}[Dokaz izreka \ref{izr:enakomerno}]
  Dokazovanja te trditve se lahko lotimo tudi takole \ldots
\end{dokaz}

\subsection{Naslov morebitnega (pod)razdelka} Besedilo naj se nadaljuje v vrstici naslova, torej za ukazom \verb|\subsection{}| ne smete izpustiti prazne vrstice.

Podobno kot lahko spremenimo ime dokaza, lahko dodamo komentar v ime trditve,
torej s pomočjo neobveznega parametra; prvega od spodnjih izpisov dosežemo z
ukazom \verb|\begin{posledica}[izrek o vmesni vrednosti]|. Če želimo v
parametru navesti vir, pri katerem bomo navedli podatek o tem, kje v viru to
trditev najdemo, pa uporabimo ukaz
%\verb|\begin{posledica}[\protect{\cite[izrek 3.14]{glob}}]|. Seveda lahko obe
možnosti kombiniramo.


\begin{posledica}[izrek o vmesni vrednosti]
  Naj bo $f$ zvezna in \ldots
\end{posledica}

Ali pa

\begin{posledica}%[izrek o vmesni vrednosti \protect{\cite[izrek 3.14]{glob}}]
  Naj bo $f$ zvezna in \ldots
\end{posledica}

Podobno lahko napovemo tudi vsebino primera.

\begin{primer}[nezvezna funkcija nima nujno lastnosti povprečne vrednosti]
  Naj bo $f \colon \R \to \R$ dana s predpisom \dots
\end{primer}



\section{Konec dela}

Na konec dela sodita angleško-slovenski slovarček strokovnih izrazov in seznam
uporabljene literature, morebitne priloge (programska koda, daljša ponovitev
dela snovi, ki je bil obravnavan med študijem \dots) pa neposredno pred ti
enoti. Slovar naj vsebuje vse pojme, ki ste jih spoznali ob pripravi dela, pa
tudi že znane pojme, ki ste jih spoznali pri izbirnih predmetih. Najprej
navedite angleški pojem (ti naj bodo urejeni po abecedi) in potem ustrezni
slovenski prevod; zaželeno je, da temu sledi tudi opis pojma, lahko komentar
ali pojasnilo. Slovarska gesla navajajte z ukazom \verb|\geslo{}{}|, npr.\
\verb|\geslo{continuous}{zvezen}|.

Pri navajanju literature si pomagajte s spodnjimi primeri; najprej je opisano
pravilo za vsak tip vira, nato so podani primeri. Člen literature napišete z
ukazom \verb|\bibitem{oznaka} podatki o viru|, kjer mora \emph{ozmaka} enolično
določati vir.  Posebej opozarjam, da spletni viri uporabljajo paket url, ki je
vključen v~.cls datoteki. Polje ``ogled'' pri spletnih virih je obvezno; če je
kak podatek neznan, ustrezno ``polje'' seveda izpustimo. Literaturo je potrebno
urediti po abecednem vrstnem redu; najprej navedemo vse vire z znanimi avtorji
(tiskane in spletne) po abecednem redu avtorjev (po priimkih, nato imenih),
nato pa spletne vire brez avtorjev, urejene po naslovih strani. Če isti vir
navajamo v dveh oblikah, kot tiskani in spletni vir, najprej navedemo tiskani
vir, nato pa še podatek o tem, kje je dostopen v elektronski obliki.

\end{document}
