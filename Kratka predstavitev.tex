\documentclass[a4paper,12pt]{article}

\usepackage[slovene]{babel}
\usepackage{amsfonts,amssymb,amsmath}
\usepackage[utf8]{inputenc}
\usepackage[T1]{fontenc}
\usepackage{lmodern}
\usepackage{graphicx}


\def\N{\mathbb{N}} % mnozica naravnih stevil
\def\Z{\mathbb{Z}} % mnozica celih stevil
\def\Q{\mathbb{Q}} % mnozica racionalnih stevil
\def\R{\mathbb{R}} % mnozica realnih stevil
\def\C{\mathbb{C}} % mnozica kompleksnih stevil
\newcommand{\geslo}[2]{\noindent\textbf{#1} \quad \hangindent=1cm #2\\[-1pc]}

\def\qed{$\hfill\Box$}   % konec dokaza
\def\qedm{$\qquad\Box$}   % konec dokaza v matematičnem načinu
\newtheorem{izrek}{Izrek}
\newtheorem{trditev}{Trditev}
\newtheorem{posledica}{Posledica}
\newtheorem{lema}{Lema}
\newtheorem{opomba}{Opomba}
\newtheorem{definicija}{Definicija}
\newtheorem{zgled}{Zgled}

\title{Metrična dimenzija grafa deliteljev niča \\ 
\Large Kratka predstavitev}
\author{Tadeja Možina}
\date{4.\ december 2023}

\begin{document}


%%%%%%%%%%%%%%%%%%%%%%%%%%%%%%%%%%%%%%%%%%%%%%%%%%%%%%%%%%%%%%%%%%%%%


\maketitle


%%%%%%%%%%%%%%%%%%%%%%%%%%%%%%%%%%%%%%%%%%%%%%%%%%%%%%%%%%%%%%%%%%%%%
\section{Uvod}
Pojem metrične dimenzije grafa je v sedemdesetih letih prejšnjega stoletja 
uvedel Peter J. Slater, uporablja pa se 
na različnih področjih, kot na primer v farmacevtski kemiji, navigaciji 
robotov in kombinatorični optimizaciji. 
Formule za metrično dimenzijo 
poljubnega grafa ni, v diplomski nalogi pa bomo navedeli 
različne izreke, ki jo bodo vseeno lahko omejili. Gledali bomo neusmerjene 
enostavne grafe, predvsem pa grafe deliteljev niča poljubnega kolobarja, 
posebej pa bomo pogledali metrično dimenzijo kolobarja ostankov po danem modulu 
in kolobarja matrik nad danim poljem.
%
\begin{definicija}
    Naj bo R kolobar in $Z(R)$ njegova množica deliteljev niča.
    \emph{Graf deliteljev niča kolobarja R} je enostaven neusmerjen graf z množico
    vozlišč $Z^{*}(R) = Z(R)\setminus\{0\} $, kjer sta dve različni vozlišči $a,b \in Z^{*}(R) $
    povezani natanko tedaj, ko je $ab = 0$ ali $ba = 0$. Označimo ga z $\Gamma(R)$.
\end{definicija}
%
\begin{zgled}
    Poglejmo si graf deliteljev niča kolobarja $\Z_{10}$. Množica njegovih vozlišč je 
    $V(\Gamma(\Z_{10})) = Z^{*}(\Z_{10}) = \{2, 4, 5, 6, 8\}$, množica povezav pa 
    je enaka $E(\Gamma(\Z_{10})) = \{2 \sim 5, 4 \sim 5, 6 \sim 5, 8 \sim 5\}$.
\end{zgled}
%
\begin{definicija}
    Naj bo $W = \{ w_1,w_2, \ldots, w_k \}$ urejena podmnožica množice $V(\Gamma)$ in 
    $v \in V(\Gamma)$. Potem 
    se $k$-dimenzionalni vektor \\
    $r(v|W)=( d(v,w_1), \ldots, d(v,w_k) )$ imenuje 
    \emph{predstavitev $v$ glede na množico $W$}. 

\end{definicija}
%
\begin{zgled}\label{prim1}
    Poglejmo si graf deliteljev niča kolobarja $\Z_{10}$. Za $W$ vzemimo 
    $\{2,4,5\}$ in poglejmo predstavitve glede na $W$. To so ... (na sliki).
    %$r(2|W)=( d(2,2), d(2,4), d(6,5) ) = ( 0, 2, 1 )$,
    %$r(4|W)=( d(4,2), d(4,4), d(4,5) ) = ( 2, 0, 1 )$,
    %$r(5|W)=( d(5,2), d(5,4), d(5,5) ) = ( 2, 2, 0 )$,
    %$r(6|W)=( d(6,2), d(6,4), d(6,5) ) = ( 2, 2, 1 )$ in 
    %$r(8|W)=( d(8,2), d(8,4), d(8,5) ) = ( 2, 2, 1 )$.
\end{zgled}
%
  Množici $W$ pravimo \emph{rešljiva množica grafa $\Gamma$}, če imata poljubni različni 
  vozlišči v $V(\Gamma)$ različni predstavitvi glede na $W$. Če je $W$ rešljiva množica 
  z najmanjšo močjo, ji pravimo \emph{baza $\Gamma$}.
%
\begin{definicija}
    \emph{Metrična dimenzija grafa $\Gamma$} je moč njegove baze. Označimo jo z $dim(\Gamma)$.
\end{definicija}
%
\begin{zgled}
    Pri prejšnjem zgledu opazimo, da $W = \{2,4,5\}$ ni rešljiva množica grafa $\Gamma(\Z_{10})$. 
    Vendar pa to ne pomeni, da rešljiva množica moči $3$ ne obstaja. Vzemimo $W_2 = \{2,4,6\}$ in 
    poglejmo predstavitve glede na $W_2$.\\
    Izkaže se, da $W_2 = \{2,4,6\}$ ni samo rešljiva množica grafa $\Gamma(\Z_{10})$, ampak 
    celo njegova baza. Baza grafa ni nujno enolično določena, v tem primeru bi lahko vzeli 
    tudi $\{2,4,8\}$, $\{2,6,8\}$ ali $\{4,6,8\}$. Metrična dimenzija grafa $\Gamma(\Z_{10})$ je 
    tako $3$.\\
    Opazimo še: če preverjamo, ali je $W= \{w_1,w_2, \ldots, w_k\} \subseteq V(\Gamma)$ 
    rešljiva množica $\Gamma$, je dovolj gledati predstavitve elementov iz 
    $V(\Gamma) \setminus W$ glede na $W$. Namreč če izberemo $w_i \in W$ bo vektor 
    $(d(w_i,w_1),d(w_i,w_2), \ldots d(w_i,w_k))$ na $i$-tem mestu imel 0. 
    Če bi katerokoli drugo vozlišče $u$ imelo enako predstavitev kot $w_i$, bi 
    potem moralo veljati, da $d(w_i,w_i) = d(u,w_i)$, kar pa je edino možno, 
    če je $u = w_i$.
\end{zgled}
%


%
%
\section{Kolobar ostankov}
Oglejmo si sedaj kolobar ostankov. Metrično dimenzijo $\Gamma(\Z_{n})$ poiščemo 
s pomočjo praštevilskega razcepa $n$. Če je $n$ praštevilo, je graf deliteljev 
niča od $\Z_{n}$ prazen graf, torej nas ta ne zanima.
%
\begin{trditev}
    Naj bo $n$ produkt dveh različnih praštevil, $n = p_1 \cdot p_2$. Potem je 
    $dim(\Gamma(\Z_{n})) = p_1 + p_2 - 4$.
\end{trditev}
%
\begin{zgled}
    $10 = 2\cdot5$, torej je $dim(\Gamma(\Z_{10})) = 5 + 2 - 4 = 3$, kar je enako kot 
    smo prej povedali.
\end{zgled}
%
\begin{trditev}
    Naj bo $p$ praštevilo.
    \begin{enumerate}
        \item Če je $n = p^2$ in $p > 2$, potem je $dim(\Gamma(\Z_{n})) = p - 2$,
        \item Če je $n = p^k$ in $k \geq 3$, potem je $dim(\Gamma(\Z_{n})) = p^{k-1} - k$.
    \end{enumerate}
\end{trditev}
%
\begin{izrek}
    Naj bo $n = \prod_{i = 1}^{r}p_i^{n_i}$ praštevilski razcep $n$. Če je $r \geq 2$ in 
    $\sum_{i=1}^{r}n_i \geq 3$, potem je 
    \begin{equation*}
        dim(\Gamma(\Z_{n})) = n - \varphi(n) - \prod_{i = 1}^{r}(n_i + 1) + 1,
    \end{equation*}
    kjer je $\varphi(n)$ Eulerjeva funkcija $\varphi$.
\end{izrek}
%
\begin{zgled}
    Poiščimo metrično dimenzijo $\Gamma(\Z_{12})$. Praštevilski razcep $12$ je 
    $12 = 2^2 \cdot 3$, torej je \\
    $dim(\Gamma(\Z_{10})) = 12 - \varphi(12) - (2 + 1)\cdot(1 + 1) + 1 = 12 - 4 - 6 + 1 = 3$.
    Za bazo bi lahko vzeli množico $W = \{2,3,4\}$.
\end{zgled}
%
%
\section{Kolobar matrik nad poljem}
Za konec si oglejmo še metrično dimenzijo grafa deliteljev niča kolobarja matrik nad 
končnim poljem. Iz Algebre 2 že vemo, da so vsa končna polja izomorfna $\Z_{p^k}$, 
za neka $p$ in $k$, kjer je $p$ praštevilo, $k \in \N$. Z 
$M_n(K)$ označimo kolobar $n \times n$ matrik nad končnim poljem $K$ ter z $Mat_n^r(K)$ 
$n \times n$ matrike ranga $r$ nad končnim poljem $K$. Označimo še $m_k(n,r)$ kot 
moč $Mat_n^r(K)$.
%
\begin{izrek}
    Naj bo $|K| = k \geq 3$. Potem je \\
    $dim(\Gamma(M_n(K))) = k^{n^2} - \sum_{i=1}^{n}\frac{m_k(n,i)}{m_k(i,i)} - m_k(n,n)$.
\end{izrek}
%
\begin{izrek}
    Naj bo $n \geq 3$ in $|K| = k = 2$. Potem je \\
    $dim(\Gamma(M_n(K))) = 2^{n^2} - \sum_{i=1}^{n}\frac{m_2(n,i)}{m_2(i,i)} - m_2(n,n)$.
\end{izrek}
%
\begin{trditev}
    $dim(\Gamma(M_2(\Z_2))) = 3$.
\end{trditev}
%
\begin{zgled}
    $\Gamma(M_2(\Z_2))$.
\end{zgled}
%
\end{document}